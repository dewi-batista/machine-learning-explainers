%%%%%%%%%%%%%%%%%%%%%%%%%%%%%%%%%%%%%%%%%%%%%%%%%%%%%%%%%%%%%%%%%%%%%%%%%%%%%%%
% Structured as three parts: Packages, Commands and Miscellaneous
%%%%%%%%%%%%%%%%%%%%%%%%%%%%%%%%%%%%%%%%%%%%%%%%%%%%%%%%%%%%%%%%%%%%%%%%%%%%%%%

%%%=== PART 1: Packages
\usepackage{amsfonts, amsmath, amssymb}       %% for general notation
\usepackage[toc,page]{appendix}
\usepackage{bbm}                              %% for indicator function notation
\usepackage{enumitem}                         %% for customising bullet point lists
\usepackage{float, graphicx, svg, subcaption} %% for figures
\usepackage{hyperref}                         %% for in-document references
\usepackage[utf8]{inputenc}                   %% for compiler to interpret .tex file
\usepackage{setspace}                         %% for setting vspace between sentences
\usepackage{tcolorbox}                        %% for tcolorboxes
\usepackage{xcolor}                           %% for pre-set colours

\usepackage[pass]{geometry} %% for the minimisation of margins
\newlength\DX
\DX=3.0in
\paperwidth=\dimexpr\paperwidth-\DX\relax
\hoffset=\dimexpr\hoffset-.5\DX\relax
\newlength\DY
\DY=2.8in
\paperheight=\dimexpr\paperheight-\DY\relax
\voffset=\dimexpr\voffset-.5\DY-.5\footskip\relax

%%%=== PART 2: Commands
\newcommand{\D}{\mathcal{D}}
\newcommand{\E}{\mathbb{E}}
\newcommand{\R}{\mathbb{R}}
\newcommand{\X}{\mathbf{X}}
\newcommand{\Z}{\mathbf{Z}}
\newcommand{\x}{\mathbf{x}}
\newcommand{\z}{\mathbf{z}}
\newcommand{\TODO}[1]{\textcolor{red}{\textbf{TODO}#1}}

\renewcommand{\l}{\left}
\renewcommand{\r}{\right}
\renewcommand{\sectionautorefname}{Section}
\renewcommand{\subsectionautorefname}{Subsection}
\renewcommand{\subsubsectionautorefname}{Subsubsection}

%%%=== PART 3: Miscellaneous
\definecolor{myDarkBlue} {RGB}{ 15, 61,138}
\definecolor{myGreen}    {RGB}{  0,128,128}
\definecolor{myLightBlue}{RGB}{173,216,230}
\definecolor{myLightRed} {RGB}{255, 99, 71}
\definecolor{myYellow}   {RGB}{255,165,  0}

\captionsetup{labelfont=bf} %% for bold caption labels

\DeclareMathOperator*{\argmax}{arg\,max} %% for argmax/argmin notation
\DeclareMathOperator*{\argmin}{arg\,min}
