%% generic
\usepackage{amsfonts, amsmath, amssymb}
\usepackage[toc,page]{appendix}
% \usepackage{geometry}
\usepackage{hyperref}
\usepackage[utf8]{inputenc}
\usepackage{setspace}

\usepackage[pass]{geometry}
\newlength\DX
\DX=3.0in
\paperwidth=\dimexpr\paperwidth-\DX\relax
\hoffset=\dimexpr\hoffset-.5\DX\relax
\newlength\DY
\DY=2.8in
\paperheight=\dimexpr\paperheight-\DY\relax
\voffset=\dimexpr\voffset-.5\DY-.5\footskip\relax

%% for the indicator function
\usepackage{bbm}

%% for figures
\usepackage{graphicx}
\usepackage{subcaption}
\usepackage{svg}

%% for displaying boxes around text
\usepackage{tcolorbox}

%% colour presets
\usepackage{xcolor}
\definecolor{c1}{RGB}{255,240,200}
\definecolor{c2}{RGB}{173,216,230}
\definecolor{c3}{RGB}{255,99,71}
\definecolor{c4}{RGB}{144,238,144}
\definecolor{c5}{RGB}{255,165,0}
\definecolor{c6}{RGB}{216,191,216}
\definecolor{c7}{RGB}{255,182,193}
\definecolor{c8}{RGB}{211,211,211}
\definecolor{c9}{RGB}{224,255,255}
\definecolor{c10}{RGB}{255,215,0}
\definecolor{c11}{RGB}{0,128,128}

%% for displaying algorithms
\usepackage[linesnumbered,ruled,vlined]{algorithm2e}
\SetKwInput{KwInput}{Input}
\SetKwInput{KwOutput}{Output}

%% for bold caption labels
\captionsetup{labelfont=bf}

%% math commands/operators
\newcommand{\D}{\mathcal{D}}
\newcommand{\Dtrain}{\mathcal{D}_{\text{train}}}
\newcommand{\Dtest}{\mathcal{D}_{\text{test}}}
\newcommand{\E}{\mathbb{E}}
\newcommand{\N}{\mathcal{N}}
\newcommand{\OX}{\Omega_{\X}}
\newcommand{\OY}{\Omega_{Y}}
\renewcommand{\P}{\mathbb{P}}
\newcommand{\R}{\mathbb{R}}
\newcommand{\T}{\text{T}}
\newcommand{\X}{\mathbf{X}}
\newcommand{\Y}{\mathbf{Y}}
\newcommand{\Z}{\mathbf{Z}}

\newcommand{\thetaOpt}{\theta^{*}}
\newcommand{\w}{\mathbf{w}}
\newcommand{\x}{\mathbf{x}}
\newcommand{\z}{\mathbf{z}}

\renewcommand{\l}{\left}
\renewcommand{\r}{\right}

\newcommand{\HRule}{\noindent\hspace*{\fill}\rule{0.95\linewidth}{0.1pt}\hspace*{\fill}}

\newtheorem{example}{Example}[section]

%% for numeric footnote markers
\renewcommand{\thefootnote}{\arabic{footnote}}

%% for argmax and argmin notation
\DeclareMathOperator*{\argmax}{arg\,max}
\DeclareMathOperator*{\argmin}{arg\,min}